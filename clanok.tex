\documentclass[10pt,twoside,slovak,a4paper]{article}

\usepackage[slovak]{babel}
%\usepackage[T1]{fontenc}
\usepackage[IL2]{fontenc}
\usepackage[utf8]{inputenc}
\usepackage{graphicx}
\usepackage{url}
\usepackage{hyperref}

\usepackage{cite}

\pagestyle{headings}

\title{Simulovanie výsledkov futbalových zápasov na základe (historických) dát}

\author{Adam Gábor\\[2pt]
	{\small Slovenská technická univerzita v Bratislave}\\
	{\small Fakulta informatiky a informačných technológií}\\
	{\small Vedúci: Ing. Fedor Lehocki}\\
	}

\date{\small 29. október 2021}

\begin{document}

\maketitle

\begin{abstract}
Odvetvie športu sa vyvíja ako každé iné a s návalom obrovského množstva informácií je čoraz populárnejšie využívanie historických dát k tvorbe pokročilých metrík a štatistík súvisiacich s hrou.

Ako tému som si vybral štúdium algoritmov, ktoré dokážu predpovedať výsledky futbalových zápasov. V mojej práci sa v stručnosti zameriam na matematiku skrytú za futbalom, za stávkovaním a na jednoduché softvéry (modely), ktoré dokážu na základe historických dát a dostupných informácií simulovať futbalové zápasy a takýmto spôsobom predpovedať ich výsledok ešte predtým, ako sa v skutočnosti budú hrať. Softvér vie na základe získaných výsledkov určiť výhodnosť či nevýhodnosť kurzov ponúkaných stávkovými kanceláriami.

\end{abstract}

\section{Úvod do problematiky}

Futbal je multimiliardový biznis. Najviac v ňom zarába ten, kto vyhráva, a preto sa tímy snažia nájsť čo najviac možností, ktoré by im zvíťaziť pomohli. A to aj mimo ihriska.

Každý futbalový tím má niekoľko dátových analytikov, ktorých jedinou úlohou je sledovať vývoj na ihrisku v číslach. Hráčom sa počítajú dotyky s loptou, nabehané kilometre, zaznamenáva sa každá strela a prihrávka. Futbalové algoritmy sú dnes schopné využiť tieto štatistiky a pomocou nich vytvoriť model simulujúci aj zápasy, ktoré sa v skutočnosti nehrali. Keďže dokážu vypočítať šancu, s akou tím skóruje, a dokážu vypočítať aj priemerný počet gólov, ktoré vsieti, môžu byť takéto modely využité na predpovedanie skutočných zápasov.

Modernou, relevantnou a často používanou metrikou je v súčasnosti ukazovateľ xG - expected goals, ktorý vyjadruje počet gólov, ktoré mal tím na základe polohy ich striel dosiahnuť. Čím bližšie je miesto vystrelenia k bráne, tým vyššia je hodnota strely v xG. Jednotlivé hodnoty sú nastavené na základe dát z miliónov už odohraných zápasov. Takýmto spôsobom napríklad vieme, že penalta má hodnotu 0.78 xG, pretože sa 78 percent penált končí gólom.

\section{Typy a zdroje použitých vstupných dát}

Keďže je futbal stále hrou, ktorú do veľkej miery ovplyvňuje ľudský faktor, je takmer nemožné s presnosťou určiť, od čoho všetkého výsledok zápasu závisí. Preto využívajú rôzne algoritmy rôznu metodológiu a rôzne vstupné dáta.

Všetky dáta by sme mohli rozdeliť do dvoch mnou vytýčených skupín - objektívne a subjektívne. Objektívne dáta sú jednoznačné čísla - napríklad počet výhier tímu alebo počet skórovaných gólov. Subjektívne dáta sa snažia brať do úvahy individuálne kvality hráčov a takisto aj spomínaný ľudský faktor a sú úzko prepojené s vedomosťami toho, kto algoritmus obsluhuje. Keď sa napríklad tímu zraní kľúčový hráč alebo musí na zápas cestovať dlhú dobu nepohodlným autobusom, pochopiteľne to ovplyvní jeho výkon.

\subsection{Forma tímu}

Pravdepodobne najzákladnejším objektívnym ukazovateľom je nedávna forma tímu. Výhodou modelu založeného na týchto dátach je to, že pracuje s reálnymi číslami popisujúcimi aktuálnu silu mužstva. Toto sa však stáva problémom v momente, keď hrá tím viacero súťaží (napríklad domácu ligu a medzinárodné poháre) a jeho výkony sa v nich rôznia \cite{dixon1997modelling}. Dobrým príkladom je slovenský klub Slovan Bratislava, ktorý síce v našej Fortuna Lige vyhráva zápas za zápasom, ale v Európskej lige sa mu proti ostatným tímom z kontinentu nedarí. Algoritmus by v prípade takýchto vstupných dát pasoval Slovan do roly favorita, a to aj napriek tomu, že hrá proti oveľa silnejšiemu zahraničnému tímu.

\subsection{Góly strelené tímom}

Väčšina algoritmov delí tímy podľa ich ofenzívnej schopnosti (t.j. toho, koľko gólov dokáže streliť) a podľa ich defenzívnej schopnosti (t.j. koľko gólov inkasuje) \cite{dixon1997modelling} \cite{razali2017predicting}. Do tejto objektívnej metriky môžu byť zaraďované aj strely či strely na bránu. Opäť sa tu ale naskytá problém - tím, ktorý inkasuje veľa gólov, môže veľa gólov aj streliť. Predstavte si situáciu, v ktorej tím vyhráva všetky zápasy 6-3. Dostane síce stále tri góly, ale strelí ích dvakrát toľko. Dobrou náhradou obrannej/útočnej sily je preto rozdiel gólov a ešte lepšou náhradou je rozdiel gólov podľa metriky xG, ktorý veľmi presne ukáže skutočnú silu porovnávaného tímu.

Poznať počty gólov, ktoré tím strieľa a obdržiava je dôležité najmä k predikcii presného výsledku - iné ukazovatele nám môžu popísať, či tím vyhrá alebo prehrá, ale nikdy nepovedia, koľko gólov dá a dostane. Na základe dát z nedávnych stretnutí vieme predikovať počet gólov tímu aj bez algoritmu - keď napríklad mužstvo strelilo v desiatich zápasoch po sebe 4 a viac gólov, je veľká šanca, že sa mu bude dariť aj v jedenástom. Predpokladaný počet gólov, ktoré tím strelí, môže byť priemerom gólov, ktoré strelil v posledných x zápasoch proti súperovi silovo a typovo podobnému nastávajúcemu súperovi.

\subsection{Relatívna sila mužstva}

\subsection{Výhoda domáceho prostredia}

Aj keď sa to na prvý pohľad zdá iba ako fáma, domáci tím má vo futbale skutočne signifikantnú výhodu. Podiel výhier domáceho k hosťovskému tímu je skoro v pomere 2:1 \cite{dixon1997modelling}. Desiatky tisíc fanúšikov povzbudzujúcich svoj obľúbený tím teda taktiež môžeme rátať ako dôležitý faktor. Pandémia COVID-19 sa stala unikátnou možnosťou na ukázanie toho, ako veľmi dokáže absencia divákov ovplyvniť výsledok futbalového zápasu \cite{2020}.

\section{Test algoritmov na reálnych zápasoch}

\section{Využitie v praxi - stávkovanie}

\section{Záver} \label{zaver} % prípadne iný variant názvu


\bibliography{literatura.bib}
\bibliographystyle{plain} % prípadne alpha, abbrv alebo hociktorý iný
\end{document}