\documentclass[10pt,twoside,slovak,a4paper]{article}
\usepackage[slovak]{babel}
\usepackage[IL2]{fontenc}
\usepackage[utf8]{inputenc}
\usepackage{graphicx}
\usepackage{url}
\usepackage{hyperref}

\usepackage{cite}

\pagestyle{headings}

\title{Simulovanie výsledkov futbalových zápasov na základe (historických) dát}

\author{Adam Gábor\\[2pt]
	{\small Slovenská technická univerzita v Bratislave}\\
	{\small Fakulta informatiky a informačných technológií}\\
	{\small Vedúci: Ing. Fedor Lehocki}\\
	}

\date{\small 29. október 2021}

\begin{document}

\maketitle

\begin{abstract}
Odvetvie športu sa vyvíja ako každé iné a s návalom obrovského množstva informácií je čoraz populárnejšie využívanie historických dát k tvorbe pokročilých metrík a štatistík súvisiacich s hrou.

Ako tému som si vybral štúdium algoritmov, ktoré dokážu predpovedať výsledky futbalových zápasov. V mojej práci sa v stručnosti zameriam na matematiku skrytú za futbalom, za stávkovaním a na jednoduché softvéry (modely), ktoré dokážu na základe historických dát a dostupných informácií simulovať futbalové zápasy a takýmto spôsobom predpovedať ich výsledok ešte predtým, ako sa v skutočnosti budú hrať. Softvér vie na základe získaných výsledkov určiť výhodnosť či nevýhodnosť kurzov ponúkaných stávkovými kanceláriami.

\end{abstract}

\section{Úvod do problematiky}

Futbal je multimiliardový biznis. Najviac v ňom zarába ten, kto vyhráva, a preto sa tímy snažia nájsť čo najviac možností, ktoré by im zvíťaziť pomohli. A to aj mimo ihriska.

Každý futbalový tím má niekoľko dátových analytikov, ktorých jedinou úlohou je sledovať vývoj na ihrisku v číslach. Hráčom sa počítajú dotyky s loptou, nabehané kilometre, zaznamenáva sa každá strela a prihrávka. Futbalové algoritmy sú dnes schopné využiť tieto štatistiky a pomocou nich vytvoriť model simulujúci aj zápasy, ktoré sa v skutočnosti nehrali. Keďže dokážu vypočítať šancu, s akou tím skóruje, a dokážu vypočítať aj priemerný počet gólov, ktoré vsieti, môžu byť takéto modely využité na predpovedanie skutočných zápasov.

Modernou, relevantnou a často používanou metrikou je v súčasnosti ukazovateľ xG - expected goals, ktorý vyjadruje počet gólov, ktoré mal tím na základe polohy ich striel dosiahnuť. Čím bližšie je miesto vystrelenia k bráne, tým vyššia je hodnota strely v xG. Jednotlivé hodnoty sú nastavené na základe dát z miliónov už odohraných zápasov. Takýmto spôsobom napríklad vieme, že penalta má hodnotu 0.78 xG, pretože sa 78 percent penált končí gólom.

Táto práca v druhej kapitole popisuje a vysvetľuje všetky možné zdroje vstupných údajov, ktoré predikujúce algoritmy spracúvajú. Delí ích na objektívne a subjektívne a určuje ich výhody a nevýhody. V tretej kapitole vyskúšame algoritmy na reálnych zápasoch, ktoré sa v priebehu týždňa hrajú, a porovnáme ich presnosť a účinnosť. Štvrtá kapitola je venovaná využitiu algoritmov v praxi - konkrétne na stávkovanie za účelom zisku.

\section{Typy a zdroje použitých vstupných dát}

Keďže je futbal stále hrou, ktorú do veľkej miery ovplyvňuje ľudský faktor, je takmer nemožné s presnosťou určiť, od čoho všetkého výsledok zápasu závisí. Preto využívajú rôzne algoritmy rôznu metodológiu a rôzne vstupné dáta.

Všetky dáta by sme mohli rozdeliť do dvoch mnou vytýčených skupín - objektívne a subjektívne. Objektívne dáta sú jednoznačné čísla - napríklad počet výhier tímu alebo počet skórovaných gólov. Subjektívne dáta sa snažia brať do úvahy individuálne kvality hráčov a takisto aj spomínaný ľudský faktor a sú úzko prepojené s vedomosťami toho, kto algoritmus obsluhuje. Keď sa napríklad tímu zraní kľúčový hráč alebo musí na zápas cestovať dlhú dobu nepohodlným autobusom, pochopiteľne to ovplyvní jeho výkon.

\subsection{Forma tímu}

Pravdepodobne najzákladnejším objektívnym ukazovateľom je nedávna forma tímu. Výhodou modelu založeného na týchto dátach je to, že pracuje s reálnymi číslami popisujúcimi aktuálnu silu mužstva. Toto sa však stáva problémom v momente, keď hrá tím viacero súťaží (napríklad domácu ligu a medzinárodné poháre) a jeho výkony sa v nich rôznia \cite{dixon1997modelling}. Dobrým príkladom je slovenský klub Slovan Bratislava, ktorý síce v našej Fortuna Lige vyhráva zápas za zápasom, ale v Európskej lige sa mu proti ostatným tímom z kontinentu nedarí. Algoritmus by v prípade takýchto vstupných dát pasoval Slovan do roly favorita, a to aj napriek tomu, že hrá proti oveľa silnejšiemu zahraničnému tímu.

\subsection{Góly strelené tímom}

Väčšina algoritmov delí tímy podľa ich ofenzívnej schopnosti (t.j. toho, koľko gólov dokáže streliť) a podľa ich defenzívnej schopnosti (t.j. koľko gólov inkasuje) \cite{dixon1997modelling} \cite{razali2017predicting}. Do tejto objektívnej metriky môžu byť zaraďované aj strely či strely na bránu. Opäť sa tu ale naskytá problém - tím, ktorý inkasuje veľa gólov, môže veľa gólov aj streliť. Predstavte si situáciu, v ktorej tím vyhráva všetky zápasy 6-3. Dostane síce stále tri góly, ale strelí ích dvakrát toľko. Dobrou náhradou obrannej/útočnej sily je preto rozdiel gólov a ešte lepšou náhradou je rozdiel gólov podľa metriky xG, ktorý veľmi presne ukáže skutočnú silu porovnávaného tímu.

Poznať počty gólov, ktoré tím strieľa a obdržiava je dôležité najmä k predikcii presného výsledku - iné ukazovatele nám môžu popísať, či tím vyhrá alebo prehrá, ale nikdy nepovedia, koľko gólov dá a dostane. Na základe dát z nedávnych stretnutí vieme predikovať počet gólov tímu aj bez algoritmu - keď napríklad mužstvo strelilo v desiatich zápasoch po sebe 4 a viac gólov, je veľká šanca, že sa mu bude dariť aj v jedenástom. Predpokladaný počet gólov, ktoré tím strelí, môže byť priemerom gólov, ktoré strelil v posledných x zápasoch proti súperovi silovo a typovo podobnému nastávajúcemu súperovi.

\subsection{Relatívna sila mužstva}

Dôležitým subjektívnym činiteľom a azda najprvoplánovejšou vecou, ktorá človeku pri simulovaní výsledku futbalového zápasu napadne, je sila oboch mužstiev. Silnejšie mužstvo má logicky vyššiu šancu na víťazstvo. Na určenie čo najpresnejšej sily mužstva je potrebné obrovské množstvo dát, a na zozbieranie obrovského množstva dát je potrebné obrovské množstvo peňazí. Presne také, aké majú k dispozícii herné spoločnosti. Legendárnu sériu FIFA od EA Sports si každý rok zahrá takmer 40 miliónov hráčov a vývojári pracujú na tom, aby bolo ich dielo čo najpodobnejšie realite. Práve preto sa stali relevantným zdrojom dát pre predikciu futbalových zápasov aj štatistiky z videohier, konkrétne hodnotenia hráčov \cite{shin2014novel}. Keď hodnotenie hráčov spriemerujeme, získavame hodnotenie tímu - podľa hry FIFA spravidla číslo od 50 do 100.

Vďaka tomuto vedia predikujúce algoritmy predpovedať aj zápasy medzi dvoma mužstvami, ktoré sa proti sebe nikdy nepostavili. Predstavme si Barcelonu hrajúcu proti Sparte Praha, pričom relatívna sila Barcelony bude 85 a relatívna sila Sparty bude 75. Nepotrebujeme vedieť, ako hrá Barcelona proti Sparte. Stačí nám vedieť, ako hrá Barcelona proti tímom hodnoteným 75 a ako hrá Sparta proti tímom hodnoteným 85. Na základe historických dát (výsledkov zápasov hraných v minulosti) dokážeme určiť aj to, koľko gólov v jednotlivých zápasoch padá. Keď napríklad hrajú dve silné mužstvá, počet strelených gólov je zvyčajne nižší ako keď hrá silný tím proti outsiderovi.

Relatívna sila tímu pomáha aj vtedy, keď klubu chýbajú dôležití hráči. Človek obsluhujúci algoritmus môže v takom prípade znížiť silu tímu na požadovanú hodnotu a získať tak presnejší výsledok simulovania, ktorý by použitím "objektívnych" dát a čísel z predošlých zápasov získať nedokázal.

\subsection{Mindset mužstva}

Pravdepodobne najsubjektívnejší ukazovateľ, ktorý nie je možné podložiť hmatateľnými dátami, je mentálne nastavenie jednotlivých tímov. Niekedy totiž tímu úplne stačí remíza a do útoku sa nehrnie, a naopak niekedy potrebuje vyhrať a útočí so všetkým, čo má. Rôzne tímy hrajú rôzne štýly futbalu a potrebujú rôzne výsledky, a práve toto berú niektoré simulujúce algoritmy do úvahy. Čím útočnejšie tím hrá, tým je väčšia šanca, že gól strelí, ale aj dostane \cite{ferrarini2014new}.

Z histórie napríklad vieme, že v dôležitých zápasoch padá menej gólov (viz. zápasy o titul majstra sveta a majstra Európy, ktoré sa v 5 z posledných 6 prípadoch rozhodli až v predĺžení), a preto je v prípade finálového alebo play-off zápasu potrebné simulátoru povedať, že budú tímy útočiť opatrnejšie a môžeme očakávať nižšie skóre.

\subsection{Výhoda domáceho prostredia}

Aj keď sa to na prvý pohľad zdá iba ako fáma, domáci tím má vo futbale skutočne signifikantnú výhodu. Podiel výhier domáceho k hosťovskému tímu je skoro v pomere 2:1 \cite{dixon1997modelling}. Desiatky tisíc fanúšikov povzbudzujúcich svoj obľúbený tím teda taktiež môžeme rátať ako dôležitý faktor. Pandémia COVID-19 sa stala unikátnou možnosťou na ukázanie toho, ako veľmi dokáže absencia divákov ovplyvniť výsledok futbalového zápasu \cite{2020}.

\section{Test algoritmov na reálnych zápasoch}

\subsection{Presnosť algoritmov podľa vstupných dát}

\subsection{Kombinovanie viacerých algoritmov naraz}

\section{Využitie v praxi - stávkovanie}

Stávkovanie je, rovnako ako futbal, multimiliardový biznis. Stávkovanie na športové zápasy funguje na kombinácii dvoch princípov - stávkový kurz (koeficient, ktorým sa v prípade úspešného tipu znásobí naša výhra) určí bookmaker - skúsený človek zodpovedný za predikciu výsledku zápasu a takisto verejnosť - keď napríklad polovica tipujúcich tipuje výhru Slovana, jeho šanca na výhru bude podľa nich 50 percent a teda kurz bude 2. Spriemerovaním kurzu bookmakera a "kurzu verejnosti" vzniká kurz, na ktorý si môžeme staviť. Kurzy sa každú chvíľu menia, väčšinou však ide o drobné zmeny.

Je dôležité uvedomiť si, že v kurzoch je zahrnutá už aj marža stávkovej kancelárie. Predstavme si napríklad hod mincou. Šanca, že sa trafíme, by mala byť pri oboch stranách 50 percent = kurz by mal byť 2, teda za každý úspešný tip získame dvojnásobok stavenej čiastky. Kurz v 50/50 situáciach sa ale pohybuje na úrovni 1.85-1.90, takže 10\% z našej výhry by si vzala stávková kancelária.

Z tohoto jasne vyplýva, že stávkovanie na 50/50 udalosti nikdy nemôže byť dlhodobo ziskové. Futbal ale rozhodne nie je 50/50 a aj bookmaker je, rovnako ako samotní hráči, iba človek, ktorý robí chyby. Algoritmom predikujúcim futbalové zápasy ani tipujúcim totiž vôbec nemusí ísť o to zistiť, ktorý tím vyhrá. Dôležité je porovnanie výsledku simulácie s pravdepodobnosťou, ktorú určí stávková kancelária. Ak je kurz vyhodený simuláciou nižší, ako kurz určený bookmakerom, znamená to, že je kurz výhodný a oplatí sa na neho staviť aj v prípade, že nenastane. Naopak, keď je síce víťaz zápasu jasný, ale stávková kancelária na neho úmyselne zníži kurz, sa staviť na takýto zápas nikdy neoplatí, aj keď si smse jeho výsledkom celkom istí.

\subsection{Prevod výsledku simulácie na pravdepodobnosť}

Keďže je výsledkom algoritmu zoznam desiatok miliónov simulovaných zápasov, nie je problém určiť, koľko z nich sa skončilo víťazstvom/prehrou jednotlivých tímov alebo remízou. To bude naša pravdepodobnosť nejakého javu. Keď touto pravdepodobnosťou vydelíme 1-tku, dostaneme decimálny, tzv. európsky kurz, s ktorým pracujú všetky tunajšie stávkové kancelárie (napr. Slovan zvíťazí v 80 percentách prípadov, tak kurz na jeho výhru je 1/0.8 = 1.25).

\subsection{Model 1X2}


\subsection{Model Over/Under a BTTS}


\section{Záver} \label{zaver}


\bibliography{literatura.bib}
\bibliographystyle{plain}
\end{document}