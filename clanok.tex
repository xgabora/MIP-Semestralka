\documentclass[10pt,twoside,slovak,a4paper]{article}

\usepackage[slovak]{babel}
%\usepackage[T1]{fontenc}
\usepackage[IL2]{fontenc}
\usepackage[utf8]{inputenc}
\usepackage{graphicx}
\usepackage{url}
\usepackage{hyperref}

\usepackage{cite}

\pagestyle{headings}

\title{Simulovanie výsledkov futbalových zápasov na základe historických dát}

\author{Adam Gábor\\[2pt]
	{\small Slovenská technická univerzita v Bratislave}\\
	{\small Fakulta informatiky a informačných technológií}\\
	{\small Vedúci: Ing. Fedor Lehocki}\\
	}

\date{\small 17. október 2021}



\begin{document}

\maketitle

\begin{abstract}
Odvetvie športu sa vyvíja ako každé iné a s návalom obrovského množstva informácií je čoraz populárnejšie využívanie historických dát k tvorbe pokročilých metrík a štatistík súvisiacich s hrou.

Ako tému som si vybral štúdium algoritmov, ktoré dokážu predpovedať výsledky futbalových zápasov. V mojej práci sa v stručnosti zameriam na matematiku skrytú za futbalom, za stávkovaním a na jednoduché softvéry (modely), ktoré dokážu na základe historických dát a dostupných informácií simulovať futbalové zápasy a takýmto spôsobom predpovedať ich výsledok ešte predtým, ako sa v skutočnosti budú hrať. Softvér vie na základe získaných výsledkov určiť výhodnosť či nevýhodnosť kurzov ponúkaných stávkovými kanceláriami.

\end{abstract}

\section{Úvod do problematiky}

Futbal je multimiliardový biznis. Najviac v ňom zarába ten, kto vyhráva, a preto sa tímy snažia nájsť čo najviac možností, ktoré by im zvíťaziť pomohli. A to aj mimo ihriska.

Každý futbalový tím má niekoľko dátových analytikov, ktorých jedinou úlohou je sledovať vývoj na ihrisku v číslach. Hráčom sa počítajú dotyky s loptou, nabehané kilometre, zaznamenáva sa každá strela a prihrávka. Futbalové algoritmy sú dnes schopné využiť tieto štatistiky a pomocou nich predpovedať, ako zápas dopadne. Dokážu vypočítať šancu, s akou tím skóruje, a dokážu vypočítať priemerný počet gólov, ktoré vsieti.

Modernou, relevantnou a často používanou metrikou je v súčasnosti ukazovateľ xG - expected goals, ktorý vyjadruje počet gólov, ktoré mal tím na základe polohy ich striel dosiahnuť. Čím bližšie je miesto vystrelenia k bráne, tým vyššia je hodnota strely v xG. Jednotlivé hodnoty sú nastavené na základe dát z miliónov už odohraných zápasov. Takýmto spôsobom napríklad vieme, že penalta má hodnotu 0.78 xG, pretože sa 78 percent penált končí gólom.

\section{Typy a zdroje použitých vstupných dát}

\section{Test algoritmu na reálnych zápasoch}

\section{Využitie v praxi - stávkovanie}

Základným problémom je teda\ldots{} Najprv sa pozrieme na nejaké vysvetlenie (časť~\ref{ina:nejake}), a potom na ešte nejaké (časť~\ref{ina:nejake}).\footnote{Niekedy môžete potrebovať aj poznámku pod čiarou.}

Môže sa zdať, že problém vlastne nejestvuje\cite{Coplien:MPD}, ale bolo dokázané, že to tak nie je~\cite{Czarnecki:Staged, Czarnecki:Progress}. Napriek tomu, aj dnes na webe narazíme na všelijaké pochybné názory\cite{PLP-Framework}. Dôležité veci možno \emph{zdôrazniť kurzívou}.


\subsection{Nejaké vysvetlenie} \label{ina:nejake}

Niekedy treba uviesť zoznam:

\begin{itemize}
\item jedna vec
\item druhá vec
	\begin{itemize}
	\item x
	\item y
	\end{itemize}
\end{itemize}

Ten istý zoznam, len číslovaný:

\begin{enumerate}
\item jedna vec
\item druhá vec
	\begin{enumerate}
	\item x
	\item y
	\end{enumerate}
\end{enumerate}


\subsection{Ešte nejaké vysvetlenie} \label{ina:este}

\paragraph{Veľmi dôležitá poznámka.}
Niekedy je potrebné nadpisom označiť odsek. Text pokračuje hneď za nadpisom.



\section{Dôležitá časť} \label{dolezita}




\section{Ešte dôležitejšia časť} \label{dolezitejsia}




\section{Záver} \label{zaver} % prípadne iný variant názvu


% týmto sa generuje zoznam literatúry z obsahu súboru literatura.bib podľa toho, na čo sa v článku odkazujete
\bibliography{literatura}
\bibliographystyle{plain} % prípadne alpha, abbrv alebo hociktorý iný
\end{document}